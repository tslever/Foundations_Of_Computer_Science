\documentclass{article}
\usepackage{amsmath}
\usepackage{graphicx} % Required for inserting images

\title{DS5012: Foundations of Computer Science\\\Large Homework: Logic, Sets, Functions and Relations}

\author{Tom Lever (tsl2b@virginia.edu)}
\date{May 20, 2025}

\begin{document}

\maketitle

QUESTIONS AND ANSWERS:\\

Q1

Given the following predicates and their meanings:

\[P(x, y): x > y\]
\[Q(x, y): x \leq y\]
\[R(x): x - 7 = 2\]
\[S(x): x > 9\]

If the universe of discourse is the real numbers, give the truth value (true
or false) of each of the following propositions:

1. \((\exists x) R(x)\). The previous statement is true because there exists a real number \(x = 9\) such that \(R(x): x - 7 = 2\) is true.

2. \((\forall y) \neg S(y)\). \(\neg S(y) = \neg (y > 9) = (y \leq 9)\). The statement at the beginning of this line is false because not all real numbers are not greater than 9, or less than or equal to 9. Consider 10, for example.

3. \((\forall x)(\exists y) P(x, y)\). The previous statement may be interpreted as, "For all real numbers \(x\) there exists a real number \(y\) such that \(x\) is greater than \(y\).". The statement at the beginning of this line is true because a real number \(y\) that is less than any real number \(x\) may always be found.

4. \((\exists y) (\forall x) Q(x, y)\). The previous statement may be interpreted as, "There exists a real number \(y\) such that for all real numbers \(x\) \(x\) is greater than \(y\).". The statement at the beginning of this line is false because for any \(y\) there exists a real number that is greater than \(y\).

5. \((\forall x) (\forall y) [P(x, y) \lor Q(x, y)]\). The previous statement may be interpreted as, "For all real numbers \(x\), for all real numbers \(y\), \(x\) is greater than \(y\) or \(x\) is less than or equal to \(y\). This statement is true because this conjunction covers all ways real numbers \(x\) and \(y\) can relate.

6. \((\exists x) [S(x) \land \neg (\forall x) R(x)]\). The previous statement may be interpreted as, "There exists a real number \(x\), such that \(x\) is greater than \(9\) and it is not the case that for all real numbers \(x\) \(x\) minus \(7\) equals \(2\). The statement "it is not the case that for all real numbers \(x\) \(x\) minus \(7\) equals \(2\)" is true. The interpretation may be rewritten as, "There exists a real number \(x\), such that \(x\) is greater than \(9\).". The statement at the beginning of this line is true because many numbers (e.g., \(10\)) satisfy the simpler interpretation.

7. \((\exists y) (\forall x) [S(y) \land Q(x, y)]\). The previous statement may be interpreted as, "There exists a real number \(y\) such that for all real numbers \(x\), \(y\) is greater than \(9\) and \(x\) is less than or equal to \(y\). The statement at the beginning of this line is false because for any \(y\) there exists a real number that is greater than \(y\).

8. \((\forall x) (\forall y) [(R(x) \land S(y)) \rightarrow Q(x, y)]\). The previous statement may be interpreted as, "For all real numbers \(x\), for all real numbers \(y\), if \(x\) - 7 equals \(2\) and \(y\) is greater than \(9\), then \(x\) is less than or equal to \(y\).". The statement "\(x\) - 7 equals \(2\)" may be rewritten as "\(x\) equals \(9\)". The interpretation may be rewritten as "For all real numbers \(x\), for all real numbers \(y\), if \(x\) equals \(9\) and \(y\) is greater than \(9\), then \(x\) is less than or equal to \(y\)." The statement at the beginning of this line is true because this interpretation is true.\\

Q2

Which of the following sentences has the logical form \(p \land q \rightarrow r\)?

1. If you don't attend the wedding, then Sam will be angry with you. The previous statement may be rewritten as, "If you don't attend the wedding and true is true, then Sam will be angry with you." The statement at the beginning of this line has the above logical form.

2. Matt is happy and so are Sam and Fae. The previous statement may be written as, "Matt is happy; [and] Sam is happy; [and] Fae is happy." The statement at the beginning of this line is not an implication and does not have the above logical form.

3. If it rains and it snows then flooding will result. The previous statement has the above logical form as "it rains" may be represented by \(p\), "it snows" may be represented by \(q\), "flooding will result" may be represented by \(r\), and the statement is conditional.

4. Students will play football or students will play soccer, but they will not attend classes. "but" may be replaced by "and". The previous statement is not an implication and does not have the above logical form.

5. Gene is smart and strong; additionally, he is a good swimmer. "; additionally, he" may be replaced by "and". The previous statement is not an implication and does not have the above logical form.\\

Q3

Which of the following formulas represents the sentence “If there are no fruit in the market then the farmers didn’t plant fruit trees or the farmers didn’t water the trees”

• p: “There are no fruit in the market”

• q: “Farmers didn’t plant fruit trees”

• r: “Farmers didn’t water the trees”

The above sentence may be represented as \(p \rightarrow (q \lor r)\), \(\neg p \lor (q \lor r)\), and \(\neg p \lor q \lor r\).

1. \(\neg p \rightarrow q = \neg (\neg p) \lor q = p \lor q\). This statement may be interpreted as, "If there are fruit in the market then farmers didn't plant fruit trees.", which neither considers that there are no fruit in the market nor that farmers didn't water the trees, as does the above sentence.

2. \(p \rightarrow (q \lor r) = \neg p \lor q \lor r\). This statement may be interpreted as, "If there are no fruit in the market then farmers didn't plant fruit trees or farmers didn't water the trees.", which represents the above sentence.

3. \((p \rightarrow q) \lor \neg r = \neg p \lor q \lor \neg r\). This statement may be interpreted as, "If there are no fruit in the market then farmers didn't plant fruit trees, or farmers watered the trees.", which is different from the above sentence.

4. \(p \rightarrow (q \lor \neg r) = \neg p \lor q \lor \neg r\). This statement may be interpreted as, "If there are no fruit in the market then farmers didn't plant fruit trees or farmers watered the trees.", which is different from the above sentence.

5. \(p \lor q \rightarrow \neg r = \neg (p \lor q) \lor \neg r = (\neg p \land \neg q) \lor \neg r\). This statement may be interpreted as, "If there are no fruit in the market or farmers didn't plant fruit trees then farmers watered the trees.", which is different from the above sentence.\\

Q4

Show that \([p \land (p \rightarrow q)] \rightarrow q\) is a tautology.

\([p \land (p \rightarrow q)] \rightarrow q\)

\([p \land (\neg p \lor q)] \rightarrow q\) by truth table

\([(p \land \neg p) \lor (p \land q)] \rightarrow q\) by distributive law

\([F \lor (p \land q)] \rightarrow q\) by complement law

\((p \land q) \rightarrow q\) by identity law

\(\neg (p \land q) \lor q\) by truth table

\((\neg p \lor \neg q) \lor q\) by De Morgan's law

\(\neg p \lor (\neg q \lor q)\) by associative law

\(\neg p \lor T\) by complement law

\(T\) by identity law

Thus, the original statement is always true and a tautology.\\

Q5

Argue that set \(A\) and its complement \(A'\) are disjoint.

A set is a collection of non-duplicate objects.

Two sets are disjoint if neither contains elements of the other.

Let set \(U\) be the universal set containing all objects \(\{x_0, x_1, ..., x_{n-1}, x_n, x_{n+1}, ...\}\).

Let set \(A\) be a set of objects \(\{x_0, x_1, ..., x_{n-1}\}\).

The complement of \(A\) \(A'\) is the set of all objects in \(U\) that are not in \(A\).

\(A' = \{x_n, x_{n+1}, ...\}\).

Because \(A'\) by definition does not contain any of the elements in \(A\), and conversely \(A\) does not contain any of the elements in \(A'\), \(A\) and \(A'\) are disjoint.\\

Q6

Which of the following is a one-to-one function?

1. \(\{(1, 2), (2, 3), (3, 4), (4, 5), (3, 7), (2, 2)\}\). A relation may be defined as a set of ordered pairs \((x, y)\). A function may be defined as a relation where a unique value \(x\) corresponds to only one value \(y\). The previous set is a relation. The previous set is not a function because the value \(2\) corresponds to both \(2\) and \(3\). The value \(3\) corresponds to both \(4\) and \(7\).

2. \(x = 5\). A function may also be defined as an equation relating \(y\) to \(x\) and constants. The previous equation does not satisfy this definition. The previous equation may also be interpreted as a set of ordered pairs \(\{(x, y) \ | \ (x = 5) \land (y \in R)\}\), which is not a function because the value \(5\) corresponds to many values of \(y\).

3. \(x = 5, 10 < y < 25\). The previous equation is similar to case 2 above, with the added constraint that values of \(y\) vary from \(10\) exclusive to \(25\) exclusive. Since the value \(5\) corresponds to many values of \(y\), the previous equation is not a function.

4. \(\{(1, 2), (2, 3), (3, 4), (2, 5), (3, 7)\}\). The previous set is similar to case 1 above. The previous set is not a function because the value \(2\) corresponds to both \(3\) and \(5\). The value \(3\) corresponds to \(4\) and \(7\).

5. \(\{(1, 2), (2, 4), (3, 6), (4, 8)\}\). The previous set is a relation. The previous set is a function given that each value of \(x\) corresponds to only one value \(y\). A function is one to one if and only if the function maps unique values of \(x\) to different values of \(y\). In other words, for all \(x_1\) and \(x_2\), if \(x_1\) and \(x_2\) are not equal, then \(y_1\) does not equal \(y_2\). Because this conditional is true, the previous function is one to one.\\

Q7

Let \(U = \{x: x \text{ is an integer and 2 } \leq x \leq 10\}\).

In each case below, determine whether \(A \subseteq B\), \(B \subseteq A\), both, or neither.

\(U = \{2, 3, 4, 5, 6, 7, 8, 9, 10\}\).

We assume that \(A\) and \(B\) are subsets of \(U\).

i. \(A = \{x: x \text{ is odd}\}\), \(B = \{x: x \text{ is a multiple of 3}\}\). \(A = \{3, 5, 7, 9\}\). \(B = \{3, 6, 9\}\). \(A\) is not a subset of \(B\) because \(5\) and \(7\) are in \(A\) and not in \(B\). \(B\) is not a subset of \(A\) because \(6\) is in \(B\) and not in \(A\). Neither \(A\) nor \(B\) is a subset of the other.

ii. \(A = \{x: x \text{ is even}\}\), \(B = \{x: x^2 \text{ is even}\}\). \(A = \{2, 4, 6, 8, 10\}\). \(B = \{2, 4, 6, 8, 10\}\). \(A\) is a subset of \(B\) because \(A\) and \(B\) are equal. \(B\) is a subset of \(A\) because \(A\) and \(B\) are equal. Both \(A\) and \(B\) are subsets of the other.

iii. \(A = \{x: x \text{ is even}\}\), \(B = \{x: x \text{ is a power of 2}\}\). \(A = \{2, 4, 6, 8, 10\}\). \(B = \{2, 4, 8\}\). \(A\) is a not a subset of \(B\) because \(6\) and \(10\) are in \(A\) and not in \(B\). \(B\) is a subset of \(A\) because \(2\), \(4\), and \(8\) are in \(A\) and \(B\).

iv. \(A = \{x: 2x + 1 > 7\}\), \(B = \{x: x^2 > 20\}\). \(A = \{x: 2x > 6\} = \{x: x > 3\} = \{4, 5, 6, 7, 8, 9, 10\}\). \(B = \{5, 6, 7, 8, 9, 10\}\). \(A\) is not a subset of \(B\) because \(4\) is in \(A\) and not in \(B\). \(B\) is a subset of \(A\) because \(5\), \(6\), \(7\), \(8\), \(9\), and \(10\) are in \(A\) and \(B\).

v. \(A = \{x: \sqrt{x} \in Z\}\), \(B = \{x: x \text{ is a power of \(2\) or \(3\)}\}\). \(A = \{4, 9\}\). \(B = \{2, 3, 4, 8, 9\}\). \(A\) is a subset of \(B\) because \(4\) and \(9\) are in \(B\). \(B\) is not a subset of \(A\) because \(2\), \(3\), and \(8\) are not in \(A\).

vi. \(A = \{x: \sqrt{x} \leq 2\}\), \(B = \{x: x \text{ is a perfect square}\}\). \(A = \{2, 3, 4\}\). \(B = \{4, 9\}\). \(A\) is not a subset of \(B\) because \(2\) and \(3\) are in \(A\) and not in \(B\). \(B\) is not a subset of \(A\) because \(9\) is in \(B\) and not in \(A\). Neither \(A\) nor \(B\) is a subset of the other.

vii. \(A = \{x: x^2 - 3x + 2 = 0\}\), \(B = \{x: x + 7 \text{ is a perfect square}\}\). \(A = \{x: (x - 1)(x - 2) = 0\} = \{x: (x = 1) \lor (x = 2)\} = \{x: x = 2\} = \{2\}\). \(B = \{2, 9\}\). \(A\) is a subset of \(B\) because \(2\) is in \(A\) and \(B\). \(B\) is not a subset of \(A\) because \(9\) is in \(B\) and not in \(A\).

\end{document}
